\documentclass{article}
\usepackage{amsmath}
\usepackage{amssymb}
\usepackage{color}
\author{GONG,Xianjin}
\title{Homework 4 of Honor Calculus}

\begin{document}
\maketitle

\section{\textcolor[rgb]{0.70,0.00,0.00}{Part \uppercase\expandafter{\romannumeral1}}}(for peer review)

\vspace{3.5mm}

\textcolor[rgb]{0.00,0.00,0.50}{\#1.4.10}\\

According to the question, we can get:\\

$x_n=l^{n-1}x_1$, and $|l|>1$\\

for $l>0$, $\lim \limits_{n \to \infty}x_n=\lim \limits_{n \to \infty}l^{n-1}\lim \limits_{n \to \infty}x_1=x_1(+\infty)=+\infty$\\

for $l<0$, for $n$ is odd, $n-1$ is even, so $\lim \limits_{n \to \infty}x_n=(-l)^{n-1}x_1=+\infty$; for $n$ is even, $n-1$ is odd, so $\lim \limits_{n \to \infty}x_n=-|l|^{n-1}x_1=-\infty$\\

$\therefore$\qquad$x_n$ diverges to $\infty$.\\

\textcolor[rgb]{0.00,0.00,0.50}{\#1.4.12}\\

According to the question, we can get:\\

$\forall n>N=\displaystyle\frac{1}{\sqrt{a}-1}$, we can get:\\

$n>\displaystyle\frac{1}{\sqrt{a}-1}$\\

$\therefore$\qquad$\displaystyle\frac{1}{n}<\sqrt{a}-1$\\

$\therefore$\qquad$\displaystyle\frac{n^2}{(n+1)^2}>\frac{1}{a}$\\

$\therefore$\qquad$\displaystyle\frac{a^{n+1}}{(n+1)^2}\frac{n^2}{a^n}>1$\\

so, $a_n$ is increasing\\

and $a_n>a_N=\displaystyle\frac{a^N}{N^2}$\\

$\therefore$\qquad It diverges to $\infty$\\

\textcolor[rgb]{0.00,0.00,0.50}{\#1.5.7(4)}\\

According to the question, we can get:\\

$\because$\qquad$\displaystyle -\left|\frac{1}{x}\right|<\left(\frac{a+x}{b}-\frac{a}{b+x}\right)<\left|\frac{2}{x}\right|\sqrt{\frac{a+x}{b}}$\\

besides, $\lim \limits_{n \to 0}\displaystyle\frac{1}{x}=+\infty$, $\lim \limits_{n \to 0}\left|\frac{2}{x}\right|\sqrt{\frac{a+x}{b}}=+\infty$\\

$\therefore$\qquad by the sandwich rule, we can get: $\lim \limits_{n \to 0}\displaystyle\left(\frac{a+x}{b}-\frac{a}{b+x}\right)=+\infty$\\

\textcolor[rgb]{0.00,0.00,0.50}{\#1.5.9(3)}\\

According to the question, we can get:\\

$\because$\qquad$\lim \limits_{n \to 0^-}x=\lim \limits_{n \to 0^+}-x=-\lim \limits_{n \to 0^+}x=0$\\

\qquad\quad$\lim \limits_{n \to 0^+}-x^2=-\lim \limits_{n \to 0^+}x^2=-\lim^2 \limits_{n \to 0^+}x=0$\\

$\therefore$\qquad$\lim \limits_{n \to 0}
\left\{
\begin{aligned}
x,\qquad\quad if\quad x<0 \\
-x^2,\quad if\quad x>0 \\
\end{aligned}
\right.=0$\\

\vspace{3.5mm}

\textcolor[rgb]{0.00,0.00,0.50}{\#1.5.13(11)}\\

According to the question, we can get:\\

$\lim \limits_{x \to 0}\displaystyle\frac{tan(sinx)}{x}=\lim \limits_{x \to 0}\frac{tan(sinx)}{sinx}\frac{sinx}{x}=\lim \limits_{sinx \to 0}\frac{tan(sinx)}{sinx}\lim \limits_{x \to 0}\frac{sinx}{x}=1$\\

\textcolor[rgb]{0.00,0.00,0.50}{Extra}\\

\begin{tabular}{|c|c||c|c||c|c|}
\hline
& value & & value & & value\\
\hline
\hline
$\displaystyle{\lim_{x \to 2^-}f(x)}$ & $2$ & $\displaystyle{\lim_{x \to 2^+}f(x)}$ & $3$ & $f(2)$ & $0$\\
\hline
$\displaystyle{\lim_{x \to -2^-}f(x)}$ & $2$ & $\displaystyle{\lim_{x \to -2^+}f(x)}$ & $-1$ & $f(-2)$ & $1$\\
\hline
$\displaystyle{\lim_{x \to 2^-}f(f(x))}$ & $2$ & $\displaystyle{\lim_{x \to 2^+}f(f(x))}$ & $f(3)$ & $f(f(2))$ & $3$\\
\hline
$\displaystyle{\lim_{x \to -2^-}f(f(x))}$ & $2$ & $\displaystyle{\lim_{x \to -2^+}f(f(x))}$ & $0$ & $f(f(-2))$ & $1$\\
\hline
$\displaystyle{\lim_{x \to 2^-}f(-f(x))}$ & $-1$ & $\displaystyle{\lim_{x \to 2^+}f(-f(x))}$ & $f(3)$ & $f(-f(2))$ & $3$\\
\hline
$\displaystyle{\lim_{x \to -2^-}f(-f(x))}$ & $-1$ & $\displaystyle{\lim_{x \to -2^+}f(-f(x))}$ & $0$ & $f(-f(-2))$ & $0$\\
\hline
\hline
\end{tabular}

\section{\textcolor[rgb]{0.70,0.00,0.00}{Part \uppercase\expandafter{\romannumeral2}}}

\textcolor[rgb]{0.00,0.00,0.50}{\#1}\\

According to the question, we can get:\\

$\displaystyle\frac{-|cos6\theta|+sin6\theta sin6h}{hcos(6(\theta+h))cos6\theta}
<\frac{sec(6(\theta+h))-sec6\theta}{h}
=\frac{soc6\theta(1-cos6h)+sin6\theta sin6h}{hcos6\theta(xos6\theta cos6h-sin6\theta sin6h)}
<\frac{|cos6\theta|+sin6\theta sin6h}{hcos(6(\theta +h))cos6\theta}$\\

by $\lim \limits_{h \to 0}\displaystyle\frac{-|cos\theta|+sin6\theta sin6h}{hcos(6(\theta+h))cos6\theta}=\infty ,
    \lim \limits_{h \to 0}\displaystyle\frac{|cos6\theta|+sin6\theta sin6h}{hcos(6(\theta +h))cos6\theta}=\infty$
    and sandwich rule, we can get:\\
    
$\lim \limits_{h \to 0}\displaystyle\frac{sec(6(\theta+h))-sec6\theta}{h}=\infty$\\

\textcolor[rgb]{0.00,0.00,0.50}{\#2}\\

(a)\\

\quad$sin3a$\\

$=\sin 2a\cos a+\cos 2a\sin a$\\

$=2\sin a\cos^2 a+\sin a-2\sin^3 a$\\

$=2\sin a(1-\sin^2 a)+\sin a-2\sin^3 a$\\

$=3\sin a-4\sin^3 a$\\

$\therefore$\qquad$\sin^3 a=\displaystyle\frac{3\sin a-\sin 3a}{4}$\\

$\therefore$\qquad$4\cdot3^{k-1}\sin^3\displaystyle\frac{x}{3^k}=\frac{4\cdot3^{k-1}\left(3\sin \frac{x}{3^k}-\frac{x}{3^{k-1}}\right)}{4}=3^k\sin\frac{x}{3^k}-3^{k-1}\sin \frac{x}{3^{k-1}}$\\

(b)\\

According to (a), we can get:\\

$3\sin \displaystyle\frac{x}{3}-\sin x=4\cdot3^0\sin^3\frac{x}{3^1}$\\

$\cdots$\\

$\displaystyle3^k\sin \frac{x}{3^k}-3^{k-1}\sin \frac{x}{3^{k-1}}=4\cdot3^{k-1}\sin^3\frac{x}{3^k}$\\

$\cdots$\\

sum them up, and denote $\displaystyle\frac{x}{3^k}=t$, we can get:\\

$\displaystyle3^k\sin\frac{x}{3^k}=\frac{x}{t}\sin t=x\frac{\sin t}{t}$\\

and $\lim \limits_{t \to \infty}x\frac{\sin t}{t}=x$\\

so we can get:\\

$\displaystyle x-\sin x=\sum \limits_{k=1}^{\infty}4\cdot3^{k-1}\sin^3\frac{x}{3^k}\neg\lim \limits_{k \to \infty}\frac{1}{6}(1-\frac{1}{9^k}x^3=\frac{1}{6}x^3$\\

(c)

According to the question, we can get:\\

$\displaystyle\sin\frac{x}{5^{k-1}}=5\sin\frac{x}{5^k}-20\sin^3\frac{x}{5^k}+16\sin^5\frac{x}{5^k}$\\

$\therefore$\qquad$\displaystyle5^k\sin\frac{x}{5^k}-5^{k-1}\sin\frac{x}{5^{k-1}}=4\cdot5^k\sin^3\frac{x}{5^k}-16\cdot5^{k-1}\sin^5\frac{x}{5^k}$\\

sum up, and we can get:\\

$x-\sin x=\sum \limits_{k=1}^{\infty}4\cdot5^k\sin^3\frac{x}{5^k}-\sum \limits_{k=1}^{\infty}16\cdot5^{k-1}\sin^5\frac{x}{5^k}$\\

also, according to (b), we can get:\\

$\sin x\geq x-\frac{1}{6}x^3$\\

$\therefore$\qquad$\displaystyle x-\sin x>\sum \limits_{k=1}^{\infty}4\cdot5^k\left[\frac{x}{5^k}-\frac{1}{6}\left(\frac{x}{5^k}\right)^3\right]^3-\sum \limits_{k=1}^{\infty}16\cdot5^{k-1}\sin^5\frac{x}{5^k}$\\

$\therefore$\qquad$\displaystyle x-\sin x>\sum \limits_{k=1}^{\infty}\left(\frac{x^3}{5^{3k}}-\frac{1}{3}\frac{x^5}{5^{5k}}+\frac{1}{36}\frac{x^7}{5^{7k}}-\frac{1}{6}\frac{x^5}{5^{5k}}+\frac{1}{18}frac{x^7}{5^{7k}}-\frac{1}{216}\frac{x^9}{5^{9k}}\right)-\sum \limits_{k=1}^{\infty}16\cdot5^{k-1}\sin^5\frac{x}{5^k}$\\

$\therefore$\qquad$\displaystyle x-\sin x>\frac{1}{6}x^3+\sum \limits_{k=1}^{\infty}\left(-\frac{1}{3}\frac{x^5}{5^{5k}}+\frac{1}{36}\frac{x^7}{5^{7k}}-\frac{1}{6}\frac{x^5}{5^{5k}}+\frac{1}{18}frac{x^7}{5^{7k}}-\frac{1}{216}\frac{x^9}{5^{9k}}\right)-\sum \limits_{k=1}^{\infty}16\cdot5^{k-1}\sin^5\frac{x}{5^k}$\\

besides, $\sum \limits_{k=1}^{\infty}\left(-\frac{1}{3}\frac{x^5}{5^{5k}}+\frac{1}{36}\frac{x^7}{5^{7k}}-\frac{1}{6}\frac{x^5}{5^{5k}}+\frac{1}{18}frac{x^7}{5^{7k}}-\frac{1}{216}\frac{x^9}{5^{9k}}\right)-\sum \limits_{k=1}^{\infty}16\cdot5^{k-1}\sin^5\frac{x}{5^k}$ can be divisible by $x^5$\\

$\therefore$\qquad $x-\sin x>\frac{1}{6}x^3+p(x)$\\

(d)\\

According to (a), (b) and (c), we can get:\\

$\displaystyle x-\sin x<\frac{1}{6}x^3$, and $\displaystyle x-\sin x>\frac{1}{6}x^3+p(x)$\\

denote $x^5q(x)=p(x)$\\

we know that:\\

$\displaystyle\frac{1}{6}+x^2q(x)<\frac{x-\sin x}{x^3}<\frac{1}{6}$\\

and by $\lim \limits_{x \to \infty}\displaystyle\frac{1}{6}+x^2q(x)=\frac{1}{6}$, $\lim \limits_{x \to \infty}\displaystyle\frac{1}{6}=\frac{1}{6}$ and sandwich rule\\

$\lim \limits_{x \to \infty}\displaystyle\frac{x-\sin x}{x^3}=\frac{1}{6}$\\

\textcolor[rgb]{0.00,0.00,0.50}{\#3}\\

(a)\\

$\displaystyle\lim \limits_{x \to \infty}\frac{1-\cos\frac{x}{2^n}}{x^2}=\frac{2\sin^2\frac{x}{2^n+1}}{x^2}$\\

denote $\displaystyle t=\frac{x}{2^n}$, we can get:\\

$\lim \limits_{t \to 0}\frac{2\sin^2t}{t^22^{2n+2}}=\frac{1}{2^{2n+1}}$\\

$\therefore$\qquad$\displaystyle\frac{1-(\cos\frac{x}{2})(\cos\frac{x}{2^2}\cdot(\cos\frac{x}{2^n})}{x^2}=\frac{(1-\cos\frac{x}{2})\cdots+(1-\cos\frac{x}{2^2})\cdots+\cdots+(1-\frac{x}{2^n})}{x^2}$\\

$\therefore$\qquad$\lim \limits_{x \to 0}\displaystyle\frac{1-(\cos\frac{x}{2})(\cos\frac{x}{2^2}\cdot(\cos\frac{x}{2^n})}{x^2}=\frac{1}{6}\left(1-\frac{1}{2^{2n+1}}\right)$\\

$\therefore$\qquad$\lim \limits_{n \to \infty}\lim \limits_{x \to 0}\displaystyle\frac{1-(\cos\frac{x}{2})(\cos\frac{x}{2^2}\cdot(\cos\frac{x}{2^n})}{x^2}=\frac{1}{6}$\\

(b)\\

Accoding to the question, we can get:\\

$\lim \limits_{n \to \infty}\displaystyle\frac{1-(\cos\frac{x}{2})(\cos\frac{x}{2^2}\cdot(\cos\frac{x}{2^n})}{x^2}=
\lim \limits_{n \to \infty}\displaystyle\frac{\sin\frac{x}{2^n}\left(1-(\cos\frac{x}{2})(\cos\frac{x}{2^2})\cdot(\cos\frac{x}{2^n})\right)}{\sin\frac{x}{2^n}x^2}
=\lim \limits_{n \to \infty}\frac{\sin\frac{x}{2^n}-\frac{1}{2^n}\sin x}{\sin\frac{x}{2^n}x^2}=\frac{1}{x^2}-\frac{\sin x}{x^3}$\\

according to question \#2, we can get:\\ 

$\therefore$\qquad$\lim \limits_{x \to 0}\lim \limits_{n \to \infty}\displaystyle\frac{1-(\cos\frac{x}{2})(\cos\frac{x}{2^2})\cdots(\cos\frac{x}{2^n})}{x^2}=\frac{1}{6}$\\

\end{document}
