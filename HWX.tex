\documentclass{article}
\usepackage{amsmath}
\usepackage{amssymb}
\usepackage{color}
\usepackage{geometry}
\usepackage{tabularx}
\usepackage{float}
\usepackage{graphicx}
\geometry{left=1.5cm}
\author{GONG,Xianjin}
\title{Homework X of Honor Calculus}

\begin{document}
\maketitle

\section{\textcolor[rgb]{0.70,0.00,0.00}{Part \uppercase\expandafter{\romannumeral1}}}(for peer or Prof. Fong or TA?)

\vspace{3.5mm}

\textcolor[rgb]{0.00,0.00,0.50}{\#1}\\

$\because$\qquad$\cos x=1-\displaystyle\frac{x^2}{2!}+o(x^2)$ and $\displaystyle(1+y)^{\frac{1}{n}}=1+\left(\binom{\frac{1}{n}}{1}\right)y+o(y)$\\

$\therefore$\qquad$\displaystyle\sqrt{\cos 2x}=1+\frac{1}{2}\cdot\frac{-4x^2}{2!}+o(x^2)$\\

\quad\qquad$\displaystyle\sqrt[3]{\cos 3x}=1+\frac{1}{3}\cdot\frac{-3^2x^2}{2!}+o(x^2)$\\

$\cdots\cdots$\\

\quad\qquad$\displaystyle\sqrt[N]{\cos Nx}=1+\frac{1}{N}\cdot\frac{-N^2x^2}{2!}+o(x^2)$\\

$\therefore$\qquad$\lim \limits_{x \to 0}\displaystyle\frac{1-(\cos x)(\sqrt{\cos 2x})(\sqrt[3]{\cos 3x})(\sqrt[4]{\cos 4x})\cdots(\sqrt[N]{\cos Nx})}{x^2}$\\

\qquad$=\lim \limits_{x \to 0}\frac{1-\left(1-\frac{x^2}{2!}-\frac{2x^2}{2!}-\frac{3x^2}{x!}-\cdots-\frac{Nx^2}{2!}\right)+o(x^2)}{x^2}=\lim \limits_{x \to 0}\frac{\frac{1}{2!}\left(\frac{N(N-1)}{2}\right)\cdot x^2+o(x^2)}{x^2}=\frac{N^2-N}{4}$\\

\textcolor[rgb]{0.00,0.00,0.50}{\#2}\\

(a)\\

We set $g(x)=f(x)-a_1x-a_2x^2-a_3x^3-\cdots-a_nx^n+o(x^n)$, then we can get:\\

$g(x)=\displaystyle\frac{1}{1-x-x^2}(x-a_1x-a_2x^2-a_3x^3-\cdots-a_nx^n+a_1x^2-a_2x^3-a_3x^4-\cdots-a_nx^{n+1}+a_1x^3-a_2x^4-a_3x^5-\cdots-a_nx^{n+2})+o(x^n)$\\

\quad$=\displaystyle\frac{1}{1-x-x^2}(-a_2x^2-\cdots-a_nx^n+a_1x^2+a_3x^3+\cdots+a_nx^n+a_nx^{n+1}+a_{n-1}x^{x+1}+a_nx^{n+2})+o(x^n)$\\

\quad$=\displaystyle\frac{1}{1-x-x^2}(a_nx^{n+1}+a_{n-1}x^{n+1}+a_nx^{n+2})+o(x^n)=\star$\\

$\because$\qquad$\lim \limits_{x \to 0}\displaystyle\frac{\star}{x^n}=\lim \limits_{x \to 0}\left[\displaystyle\frac{1}{1-x-x^2}\cdot\frac{a_nx^{n+1}+a_{n-1}x^{n+1}+a_nx^{n+2}}{x^n}+\frac{0(x^n)}{x^n}\right]=0$\\

$\therefore$\qquad$f(x)=a_1x+a_2x^2+a_3x^3+\cdots+a_nx^n+o(x^n)$ when $x \to 0$\\

(b)\\

We can get: $f(x)=\displaystyle\frac{-\frac{\frac{\sqrt{5}}{5}+1}{2}}{x-\frac{-1-\sqrt{5}}{2}}+\frac{\frac{\frac{\sqrt{5}}{5}-1}{2}}{x-\frac{-1+\sqrt{5}}{2}}$\\

and we know that for $g(x)=\frac{A}{x-\alpha}$, $g^{(n)}(x)=(-1)^n\displaystyle\frac{n!A}{(x-\alpha)^{n+1}}$\\

$\therefore$\qquad$f^{(n)}(x)=\displaystyle\frac{(-1)^{n+1}\cdot n!-\frac{\frac{\sqrt{5}}{5}+1}{2}}{\left(x-\frac{-1-\sqrt{5}}{2}\right)^{n+1}}+\frac{(-1)^{n}\cdot n!+\frac{\frac{\sqrt{5}}{5}-1}{2}}{\left(x-\frac{-1+\sqrt{5}}{2}\right)^{n+1}}$\\

$\therefore$\qquad$f^{(n)}(0)=\displaystyle\frac{(-1)^n}{2}\cdot\left(\frac{\frac{\sqrt{5}}{5}-1}{\left(\frac{1-\sqrt{5}}{2}\right)^{n+1}}-\frac{\frac{\sqrt{5}}{5}+1}{\left(\frac{1+\sqrt{5}}{2}\right)^{n+1}}\right)=\frac{(-1)^n}{\sqrt{5}}\left(\frac{1}{\left(\frac{1-\sqrt{5}}{2}\right)^n}-\frac{1}{\left(\frac{1+\sqrt{5}}{2}\right)^n}\right)=\frac{1}{\sqrt{5}}\left(\left(\frac{1+\sqrt{5}}{2}\right)^n-\left(\frac{1-\sqrt{5}}{2}\right)^n\right)$\\

\textcolor[rgb]{0.00,0.00,0.50}{\#3}\\

(a)\\

Let's set $f(x)=e^x$, then we know $f'(x)=e^x$ which is continuous in R and $f^{(2)}(x)=e^x$ which is greater than $0$ in R\\

$\therefore$\qquad$f(x)$ is a convex function.\\

$\therefore$\qquad according to Jensen's inequality, we can get: $f(a_1x_1+a_2x_2)\leq a_1f(x_1)+a_2f(x_2)$ when $a_1+a_2=1$\\

set $a_1=\displaystyle\frac{1}{p}$, $a_2=\displaystyle\frac{1}{q}$, $x_1=\log a^p$ and $x_2=\log b^q$\\

then we can get: $e^{\frac{\log a^p}{p}+\frac{\log b^q}{q}}=\displaystyle\frac{e^{\log a^p}}{p}+\frac{e^{\log b^q}}{q}$\\

$\therefore$\qquad$ab\leq\displaystyle\frac{a^p}{p}+\frac{b^q}{q}$\\

(b)\\

We set $\widetilde{x_i}=\displaystyle\frac{x_i}{||\textbf{x}||_p}, \widetilde{y_i}=\displaystyle\frac{y_i}{||\textbf{y}||_p}$\\

then, $\widetilde{x_i}\widetilde{y_i}\leq\displaystyle\frac{\widetilde{x_i}^p}{p}+\frac{\widetilde{y_i}^q}{q}$\\

$\therefore$\qquad$\displaystyle\frac{x_iy_i}{||\textbf{x}||_p||\textbf{y}||_q}\leq\frac{\frac{x_i^p}{\sum \limits_{i=1}^n|x_i|^p}}{p}+\frac{\frac{y_i^q}{\sum \limits_{i=1}^n|y_i|^q}}{q}$\\

$\therefore$\qquad$\displaystyle\frac{x_1y_1}{||\textbf{x}||_p||\textbf{y}||_q}\leq\frac{\frac{x_1^p}{\sum \limits_{i=1}^n|x_i|^p}}{p}+\frac{\frac{y_1^q}{\sum \limits_{i=1}^n|y_i|^q}}{q}$\\

$\therefore$\qquad$\displaystyle\frac{x_2y_2}{||\textbf{x}||_p||\textbf{y}||_q}\leq\frac{\frac{x_2^p}{\sum \limits_{i=1}^n|x_i|^p}}{p}+\frac{\frac{y_2^q}{\sum \limits_{i=1}^n|y_i|^q}}{q}$\\

$\therefore$\qquad$\displaystyle\frac{x_3y_3}{||\textbf{x}||_p||\textbf{y}||_q}\leq\frac{\frac{x_3^p}{\sum \limits_{i=1}^n|x_i|^p}}{p}+\frac{\frac{y_3^q}{\sum \limits_{i=1}^n|y_i|^q}}{q}$\\

$\cdots$\\

$\therefore$\qquad$\displaystyle\frac{x_ny_n}{||\textbf{x}||_p||\textbf{y}||_q}\leq\frac{\frac{x_n^p}{\sum \limits_{i=1}^n|x_i|^p}}{p}+\frac{\frac{y_n^q}{\sum \limits_{i=1}^n|y_i|^q}}{q}$\\

sum them up\\

$\therefore$\qquad$\displaystyle\frac{x_1y_1+x_2y_2+\cdots+x_Ny_N}{||\textbf{x}||_p||\textbf{y}||_q}\leq\frac{x_1^p+x_2^p+\cdots+x_N^p}{p\sum \limits_{i=1}^N|x_i|^p}+\frac{y_1^q+y_2^q+\cdots+y_N^q}{q\sum \limits_{i=1}^N|y_i|^q}<\frac{|x_1|^p+|x_2|^p+\cdots+|x_N|^p}{p\sum \limits_{i=1}^N|x_i|^p}+\frac{|y_1|^q+|y_2|^q+\cdots+|y_N|^q}{q\sum \limits_{i=1}^N|y_i|^q}=\frac{1}{p}+\frac{1}{q}=1$\\

$\therefore$\qquad$\textbf{x}\cdot\textbf{y}\leq||\textbf{x}||_p||\textbf{y}||_q$\\

(c)\\

According to the question, we can get:\\

$||\textbf{x}+\textbf{y}||_p^p=|x_1+y_1|^p+|x_2+y_2|^p+\cdots+|x_n+y_n|^p$\\

\qquad\qquad$=|x_1+y_1||x_1+y_1|^{p-1}+|x_2+y_2||x_2+y_2|^{p-1}+\cdots+|x_n+y_n||x_n+y_n|^{p-1}$\\

\qquad\qquad$\leq(||\textbf{x}||_p+||\textbf{y}||_p)(|x_1+y_1|^{(p-1)q}+|x_2+y_2|^{(p-1)q}+\cdot+|x_n+y_n|^{(p-1)q})^{\frac{1}{q}})$\\

$\because$\qquad$\displaystyle\frac{1}{p}+\frac{1}{q}=1$\\

$\therefore$\qquad$||\textbf{x}+\textbf{y}||_p^p\leq(||\textbf{x}||_p+||\textbf{y}||_p)(|x_1+y_1|^{(p-1)\left(\frac{p}{p-1}\right)}+|x_2+y_2|^{(p-1)\left(\frac{p}{p-1}\right)}+\cdot+|x_n+y_n|^{(p-1)\left(\frac{p}{p-1}\right)})^{\frac{1}{q}})$\\

\qquad\qquad=$(||\textbf{x}||_p+||\textbf{y}||_p)\displaystyle\frac{||\textbf{x}+\textbf{y}||^p_p}{||\textbf{x}+\textbf{y}||_p}$\\

$\therefore$\qquad$||\textbf{x}+\textbf{y}||_p\leq||\textbf{x}||_p+||\textbf{y}||_p$\\

(d)\\

set $\max\{|x_1|, |x_2|, \cdots, |x_n|\}=A$ and the n that we get from $||x||_p$ is actually a finite real number\\

then $A\leq\sum \limits_{i=1}^n|x_i|^p\leq n\cdot A$\\

$\because$\qquad$\lim \limits_{p \to \infty}A=A$ and $\lim \limits_{p \to \infty}\sqrt[p]{n}A=A$\\

$\therefore$\qquad according to sandwich rule, $\lim \limits_{p \to \infty}\left(\sum \limits_{i=1}^n|x_i|^p\right)^{\frac{1}{p}}=A$\\

although the original one is not in the form of limits, but the motivation comes from the limit of it as p goes to infinity.\\

\end{document}
