\documentclass{article}
\usepackage{amsmath}
\usepackage{amssymb}
\usepackage{color}
\author{GONG,Xianjin}
\title{Homework 6 of Honor Calculus}

\begin{document}
\maketitle

\section{\textcolor[rgb]{0.70,0.00,0.00}{Part \uppercase\expandafter{\romannumeral1}}}(for peer review)

\vspace{3.5mm}

\textcolor[rgb]{0.00,0.00,0.50}{\#1.6.17}\\

To prove $\lim \limits_{x \to 0}a^x=1$, by composition rule, it is the same as to prove $\displaystyle\lim \limits_{x \to \infty}a^{\frac{1}{x}}$\\

and we know that $[x]<x<[x]+1$, so $\displaystyle\frac{1}{[x]+1}<\frac{1}{x}<\frac{1}{[x]}$\\

when $a=1$, for sure $\displaystyle\lim \limits_{x \to \infty}a^{\frac{1}{x}}=1$\\

when $a>1$, we can get $\displaystyle a^{\frac{1}{[x]+1}}<a^{\frac{1}{x}}<a^{\frac{1}{[x]}}$\\

besides, $\forall\epsilon>0, \exists N_1$ s.t. $\displaystyle n>N_1\Rightarrow\left|a^{\frac{1}{n}}-1\right|<\epsilon$\\

so, $\forall\epsilon>0, \exists N_2=N_1+1023$ s.t. $\displaystyle [x]>N_2\Rightarrow\left|a^{\frac{1}{[x]+1}}-1\right|<\epsilon,  \left|a^{\frac{1}{[x]}}-1\right|<\epsilon$\\

$\therefore$\qquad$\displaystyle 1-\epsilon<a^{\frac{1}{[x]+1}}<a^{\frac{1}{x}}<a^{\frac{1}{[x]}}<1+\epsilon$\\

$\therefore$\qquad$\displaystyle \left|a^{\frac{1}{x}}-1\right|<\epsilon$\\

$\therefore$\qquad$\displaystyle \lim \limits_{x \to 0}a^x=\lim \limits_{x \to \infty}a^{\frac{1}{x}}=1$\\

when $a<1$, $\lim \limits_{x \to 0}a^x=\lim \limits_{x \to 0}\displaystyle\frac{1}{a^x}=\frac{1}{1}=1$\\

$\therefore$\qquad$\lim \limits_{x \to 0}a^x=1$\\

\textcolor[rgb]{0.00,0.00,0.50}{\#1.6.23(8)}\\

$\because$\qquad$\lim \limits_{n \to \infty}\displaystyle\left(1+\frac{1}{n}\right)^{\frac{n^2}{n+1}}=\lim \limits_{n \to \infty}\left[\left(1+\frac{1}{n}\right)^n\right]^{\frac{n}{n+1}}$\\

\qquad\quad and $\lim \limits_{n \to \infty}\displaystyle\frac{n}{n+1}=\lim \limits_{n \to \infty}\frac{1}{1+\frac{1}{n}}=\frac{1}{1+0}=1$\\

\qquad\quad and $\lim \limits_{n \to \infty}\displaystyle\left(1+\frac{1}{n}\right)^n=e$\\

$\therefore$\qquad$\lim \limits_{n \to \infty}\left(1+\frac{1}{n}\right)^{\frac{n^2}{n+1}}=e^1=e$\\

\textcolor[rgb]{0.00,0.00,0.50}{\#1.7.2}\\

$f(x)=(x-1)(x-2)(x-3)\cdot D(x)$\\

when\\

$$D(x)=
\begin{cases}
0& \text{$x\notin Q$}\\
1& \text{$x\in Q$}
\end{cases}$$

\textcolor[rgb]{0.00,0.00,0.50}{\#1.7.11(2)(5)}\\

(2)\\

we know that:$\lim \limits_{x \to 0}\log\left(\displaystyle\frac{ax+b}{cx+d}\right)=\log{\displaystyle\frac{b}{d}}$\\

let's set this limit as L\\

besides, $\lim \limits_{x \to 0}\frac{1}{x}=+\infty$\\

so when $b>d$, $L>0\Rightarrow\lim \limits_{x \to 0}\displaystyle\frac{1}{x}\log\left(\displaystyle\frac{ax+b}{cx+d}\right)=+\infty*L=+\infty$\\

when $b<d$, $L<0\Rightarrow\lim \limits_{x \to 0}\displaystyle\frac{1}{x}\log\left(\displaystyle\frac{ax+b}{cx+d}\right)=+\infty*L=-\infty$\\

when$b=d$\\

$\lim \limits_{x \to 0}\log\left(\displaystyle\frac{ax+b}{cx+d}\right)^{\displaystyle\frac{1}{x}}=\lim \limits_{x \to \infty}\log\left(1+\frac{a-c}{c+dx}\right)^x$\\

set $\displaystyle\frac{1}{y}=\frac{a-c}{c+dx}$, then $x=\displaystyle\frac{(a-c)y-c}{d}$\\

then  $\displaystyle\lim \limits_{x \to \infty}\log\left(1+\frac{a-c}{c+dx}\right)^x=\lim \limits_{y \to \infty}\log\left[\left(1+\frac{1}{y}\right)^y\right]^{\frac{(a-c)y-c}{dy}}=\lim \limits_{y \to \infty}\frac{(a-c)y-c}{dy}=\frac{a-c}{d}$\\

(5)\\

by composition rule, it is just the same as (2), only switch a with b, c with d.\\

\textcolor[rgb]{0.00,0.00,0.50}{\#1.7.13(2)}\\

Let's set $y=a^x-1$, then $x=\log_a(y+1)$\\

so, we can get: $\lim \limits_{y \to 0}\displaystyle\frac{y}{\log_a(y+1)}=\lim \limits_{y \to 0}\frac{y}{\frac{\log(y+1)}{\log a}}=\log a$\\

\section{\textcolor[rgb]{0.70,0.00,0.00}{Part \uppercase\expandafter{\romannumeral2}}}(for TA)

\vspace{3.5mm}

\textcolor[rgb]{0.00,0.00,0.50}{\#1}\\

For sequences $\{x_n\}, \{y_n\}$ that $x_n=\displaystyle\frac{1}{2n\pi-\frac{\pi}{2}}, y_n=\displaystyle\frac{1}{2n\pi+\frac{\pi}{2}}$\\

we can get both $\{x_n\}$ and $\{y_n\}$ converge to $0$\\

but $f(x_n)=-1$, $f(y_n)=1$\\

so $\{f(x_n)\}$ and $\{f(y_n)\}$ converge to different limits\\

this means that:\\

if we assume $\lim \limits_{x \to 0}f(x)$ exists, and it is $L$\\

we choose $\epsilon=\displaystyle\frac{1}{2}$, than $\exists\delta$ s.t. $|x|<\delta\Rightarrow|f(x)-L|<\epsilon$\\

when $n$ is sufficiently large, both $x_n$, and $y_n$ can be less than any $\delta$\\

so $|f(x_n)-L|+|f(y_n)-L|<1$\\

but $|f(x_n)-L|+|f(y_n)-L|>|f(x_n)-L-f(y_n)-L|=2$\\

so, by contradiction, $\lim \limits_{x \to 0}f(x)$ doesn't exists.\\

\textcolor[rgb]{0.00,0.00,0.50}{\#2}\\

In set $R$, for any interval $(\alpha,\beta)$ that $1<\alpha<\beta$, we can get a new interval $\displaystyle(\frac{\beta}{\alpha},\beta)$ that $1<\displaystyle\frac{\beta}{\alpha}<\beta$\\

$\because$\qquad$\lim \limits_{n \to \infty}\displaystyle\frac{a_{n+1}}{a_n}=1$\\

so we can choose $\epsilon=\displaystyle\frac{\beta}{\alpha}-1$, s.t. $\exists N_1$, $n>N_1\Rightarrow\left|\displaystyle\frac{a_{n+1}}{a_n}-1\right|<\epsilon$\\

besides, $\{a_n\}$ is an strictly increasing sequence and $\lim \limits_{n \to \infty}a_n=+\infty$\\

$\therefore$\qquad$1<\displaystyle\frac{a_{n+1}}{a_n}<\frac{\beta}{\alpha}$ and $\displaystyle\frac{a_n}{a_{n_0}}>\beta$ for sufficiently large
$ n>N_2>n_0$\\

so, for $n>N_2+N_1+1023^{1023}$, $\displaystyle\frac{a_{n+1}}{a_n}<1<\frac{\beta}{\alpha}<\beta<\frac{a_n}{a_{n_0}}$\\

therefore there must be some numbers(we can them $x_n$) that can be written as $\displaystyle\frac{a_m}{a_n}$ in $(\displaystyle\frac{\beta}{\alpha},\beta)$, and there will also be other numbers(we call them $y_n$) that can't be written as $\displaystyle\frac{a_m}{a_n}$ in $(\displaystyle\frac{\beta}{\alpha},\beta)$\\

$\therefore$\qquad$\lim \limits_{x \to c}f(x)=1$ when $c=x_n$, $\lim \limits_{x \to c}f(x)=0$ when $c=y_n$\\

and $1\neq0$, so $\lim \limits_{x \to c}f(x)$ doesn't exists\\

\textcolor[rgb]{0.00,0.00,0.50}{\#3}\\

(a)\\

$\because$\qquad$f$ is a continuous function, so $\lim \limits_{x \to c}f(x)=f(c)$\\

$\therefore$\qquad$\lim \limits_{x \to c}f(x)-x=f(c)-c$, so $f(x)-x$ is also a continuous function\\

consider different situation for $f(0)-0$, and $f(1)-1$\\

when $f(0)-0=0$ or $f(1)-1=0$, $1$ or $2$ can fulfill the requirment\\

when $f(0)-0>0$ and $f(1)-1<0$, according to intermediate value theorem, there exists $x_0\in(0,1)$ s.t. $f(x_0)-x_0=0$\\

$\therefore$\qquad there must be a $x_0$ s.t. $f(x_0)=x_0$\\

(b)\\

the same as (a), $g(x+1)-g(x)$ is also a continuous function\\

consider different situation for $g(1)-g(0)$ and $g(2)-g(1)$\\

when $g(1)-g(0)=0$, $g(2)-g(1)=0$ too\\

so $0$ and $1$ can fulfill the requirement\\

when $g(1)-g(0)<0$, $g(2)-g(1)>0$\\

according to intermediate value theorem, so there must be a $x_0$ s.t. $g(x_0+1)-g(x_0)=0$\\

when $g(1)-g(0)>0$, $g(2)-g(1)<0$\\

according to intermediate value theorem, so there also must be a $x_0$ s.t. $g(x_0+1)-g(x_0)=0$\\

$\therefore$\qquad there must be a $x_0$ s.t. $g(x_0+1)=g(x_0)$\\

(c)\\

we can prove this by contradiction\\

if $h(x)$ is not a constant function\\

there must be some $x_1$, $x_2$ s.t. $h(x_1)<h(x_2)$ and they are both irrational number\\

but between any two irrational number, there must a rational number, we set a rational number $q$ s.t $h(x_1)<q<h(x_2)$\\

so according to intermediate value theorem, we can know that there must exists a $x_1<x_3<x_2$ s.t $h(x_3)=q$\\

but $h(x)$ is irrational for any $x\in R$\\

so by contradiction, we can get $h(x)$ is a continuous function\\

\textcolor[rgb]{0.00,0.00,0.50}{\#4}\\

(a)\\

According to the question, we can know that:\\

$\forall\epsilon>0, \exists\delta=\displaystyle\frac{1023}{1013}\epsilon$ s.t. $0<|x-a|<\delta\Rightarrow-\delta+a<x<\delta+a$\\

$\therefore$\qquad$|f(x)-f(a)|<\displaystyle\frac{1013}{1023}|x-a|<\epsilon$\\

$\therefore$\qquad$\lim \limits_{x \to a}f(x)=f(a)$\\

$\therefore$\qquad$f(x)$ is continuous\\

(b)\\

$\because$\qquad$x_n=f(x-1)$\\

$\therefore$\qquad$|x_2-x_1|<|x_1-x_0|\displaystyle\frac{1013}{1023}\Rightarrow|x_{n+k+1}-x_{n+k}|<\left(\frac{1013}{1023}\right)^k|x_{n+1}-x_n|$\\

assume $0<n<m=n+k$\\

then $\forall\epsilon>0, \exists N=\log_{\frac{1013}{1023}}\displaystyle\frac{\epsilon}{200|a_1-a_0|}$ s.t. $n>N\Rightarrow$ $\displaystyle
|x_m-x_n|<\sum_{i=0}^{k-1}|a_{n+i+1}-a_{n+1}|<\sum_{i=0}^{k-1}\left(\displaystyle\frac{1013}{1023}\right)^i|a_{n+1}-a_n|=\frac{1\left(1-\frac{1013}{1023}\right)^{k-1}}{1-\frac{1013}{1023}}|a_{n+1}-a_n|<200|a_{n+1}-a_n|<200\left(\frac{1013}{1023}\right)^n|a_1-a_0|<\epsilon$\\

so $x_n$ is a cauchy sequence\\

and we can get: $0<x_n+1-x_n=f(x)-x_n<\displaystyle\frac{1013}{1023}|x_n-x_{n-1}|$\\

by sandwich rule, we can get: $\lim \limits_{x \to }[f(x)-x]=f(L)-L=0$\\

$\therefore$\qquad$f(L)=L$\\

\textcolor[rgb]{0.00,0.00,0.50}{\#5}\\

(a)\\

if $f(x)$ is not a monotone sequence, there must be $x_1<x_2<x_3$ s.t. $f(x_1)>f(x_2),f(x_3)>f(x_2)$\\

if $f(x_1)=f(x_3)$, then f(x)is not bijective\\

if $f(x_1)>f(x_3)$, then according to intermediate theorem, we can find $x'$, s.t $f(x')=f(x_3)$ meanwhile $x_1<x'<x_2$, it is not bijective either\\

if $f(x_1)<f(x_3)$, then according to intermediate theorem, we can find $x'$, s.t $f(x')=f(x_1)$ meanwhile $x_2<x'<x_3$, it is not bijective either\\

by contradiction, we can get that $f(x)$ must be monotone\\

besides, $f(0)<f(1)$, so $f(x)$ msut be monotone increasing\\

if $f(x)$ is not strictly increasing, there must be some interval $(a,b)$ s.t. $f(a)=f(b)$\\

but $f(x)$ is bijective\\

so, by contradiction, we can get $f(x)$ is strictly increasing\\

(b)\\

$\because$\qquad$\{a_n\}$ is strictly increasing sequence, $f(x)$ is strictly increasing function\\

$\therefore$\qquad$f(a_n)$ is strictly increasing, $f^{-1}(x)$ is strictly increasing too\\

$\therefore$\qquad$\displaystyle\frac{\frac{1}{n+1}\sum_{k=1}^{n+1}f(a_k)}{\frac{1}{n}\sum_{k=1}^{n}f(a_n)}=\left(1+\frac{f(a_{n+1})}{f(a_1)+\cdots+f(a_n)}\right)\frac{n}{n+1}$\\

besides, $f(a_{n+1})>f(a_n), f(a_1)+\cdots+f(a_n)<n\cdot f(a_n)$\\

therefore, $\displaystyle\frac{\frac{1}{n+1}\sum_{k=1}^{n+1}f(a_k)}{\frac{1}{n}\sum_{k=1}^{n}f(a_n)}>1$\\

therefore, $\frac{1}{n}\sum_{k=1}^{n}f(a_n)$ is strictly increasing\\

therefore, $\left\{f^{-1}(\frac{1}{n}\sum_{k=1}^{n}f(a_n))\right\}_{n=1}^\infty$ is strictly increasing\\

(c)\\

because $b_n$ is bounded, this means:\\

when it is bounded above\\

assume that $U$ is upper bound of $b_n$\\

it implies that, $b_n<U$ for any $n$\\

and because $f(x)$ is strictly increasing, so $f(b_n)<f(U)$ for any $n$\\

so, $\displaystyle\frac{1}{n}\sum_{k=1}^nf(b_k)<f(U)$\\

so, $f^{-1}(\frac{1}{n}\sum_{k=1}^{n}f(a_n))<f^{-1}(f(U))$\\

so, $\left\{f^{-1}(\frac{1}{n}\sum_{k=1}^{n}f(b_n))\right\}_{n=1}^\infty$ is bounded above\\

when it is bounded below, it is just the same as bounded above\\

so, $\left\{f^{-1}(\frac{1}{n}\sum_{k=1}^{n}f(b_n))\right\}_{n=1}^\infty$ is bounded\\ 



\end{document}
