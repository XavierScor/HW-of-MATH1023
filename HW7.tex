\documentclass{article}
\usepackage{amsmath}
\usepackage{amssymb}
\usepackage{color}
\usepackage{geometry}
\geometry{left=1.5cm}
\author{GONG,Xianjin}
\title{Homework 7 of Honor Calculus}

\begin{document}
\maketitle

\section{\textcolor[rgb]{0.70,0.00,0.00}{Part \uppercase\expandafter{\romannumeral1}}}(for peer review)

\vspace{3.5mm}

\textcolor[rgb]{0.00,0.00,0.50}{\#2.1.2(2)}\\

$\lim \limits_{x \to 0}\displaystyle\frac{\sqrt{x+9}-3}{x}=\displaystyle\frac{d(\sqrt{x+9})}{dx}\bigg|_{x=0}$\\

\textcolor[rgb]{0.00,0.00,0.50}{\#2.1.3(4)}\\

$\displaystyle\frac{d(\tan x)}{dx}\bigg|_{x=0}=\lim \limits_{h \to 0}\frac{\tan{h+0}-\tan0}{h}=\lim \limits_{h \to 0}\frac{\tan h}{h}=\lim \limits_{h \to 0}\frac{\sin h}{h}\frac{1}{\cos h}=1$\\

and because $\tan0+1{x-0}=x$, so the linear approximation should be $x$\\

\textcolor[rgb]{0.00,0.00,0.50}{\#2.1.9}\\

$\displaystyle\frac{d\left(|x|^p\right)}{dx}\bigg|_{x=0}=\lim \limits_{h \to 0}\frac{|h+0|^p-|0|^p}{h}=\frac{|h|^p}{h}=
\begin{cases}
|h|^{p-1}=0& \text{h$>$0}\\
-|h|^{p-1}=0& \text{h$<$0}
\end{cases}$\\

so, $|x|^p$ is differentiable at 0\\

\textcolor[rgb]{0.00,0.00,0.50}{\#2.1.15(4)}\\

$\displaystyle\frac{df(x)}{dx}\bigg|_{x=0^+}=\lim \limits_{h \to 0^+}\frac{\log{1+0+h}-0}{h}=\lim \limits_{h \to 0^+}\frac{\log{h+1}}{h}=\lim \limits_{h \to 0^+}\log(1+h)^{\frac{1}{h}}=1$\\

$ $\\

$\displaystyle\frac{df(x)}{dx}\bigg|_{x=0^-}=\lim \limits_{h \to 0^-}\frac{e^{0+h}-1-e^0+1}{h}=\lim \limits_{h \to 0^-}\frac{e^h-1}{h}=\lim \limits_{h \to 0^-}\frac{e^h}{h}-\lim \limits_{h \to 0^-}\frac{1}{h}=1-0=1$\\

$\therefore$\qquad it's differentiable at $0$\\

\textcolor[rgb]{0.00,0.00,0.50}{\#2.2.25}\\

$\left(x\sqrt{x^2+a}+a\log(x+\sqrt{x^2+a})\right)'$\\

$=\sqrt{x^2+a}+\sqrt{y}'|_{y=x^2+a}(x^2+a)'\cdot x+\displaystyle\frac{a}{\sqrt{x^2+a}}$\\

$=\sqrt{x^2+a}+\displaystyle\frac{1}{2\sqrt{x^2+a}}2x\cdot x+\frac{a}{\sqrt{x^2+a}}$\\

$=\sqrt{x^2+a}+\displaystyle\frac{x^2+a}{\sqrt{x^2+a}}=2\sqrt{x^2+a}$\\

and because $\sqrt{x^2+ax+b}=\sqrt{\left(x+\displaystyle\frac{a}{2}\right)^2+b-\frac{a^2}{4}}$\\

$\therefore$\qquad we set $y=x+\displaystyle\frac{a}{2}$ and $c=b-\displaystyle\frac{a^2}{4}$\\

$\therefore$\qquad we can get\qquad $\displaystyle\frac{1}{2}\left[\left(x+\frac{a}{2}\right)\sqrt{\left(x+\frac{a}{2}\right)^2+b-\frac{a^2}{4}}+\left(b-\frac{a^2}{4}\right)\log\left(x+\frac{a}{2}+\sqrt{\left(x+\frac{a}{2}\right)^2=b-\frac{a^2}{4}}\right)\right]$\\

$ $\\

\qquad\qquad\qquad\qquad$=\displaystyle\frac{1}{2}\left[y\sqrt{y^2+c}+c\log(y+\sqrt{y^2+c})\right]=\sqrt{y^2+c}=\sqrt{x^2+ax++b}$\\

\textcolor[rgb]{0.00,0.00,0.50}{Extra}\\

(1)\\

$\left(x^{x^x}\right)'=\left(e^{\log x^{x^x}}\right)'$\\

we set $y=x^x$, and during the lecture, we have already known $y'=e^{x\log x}(\log x+1)$\\

$\therefore$\qquad$\left(e^{\log x^{x^x}}\right)'=e^{\log x\cdot y}(\log x\cdot y)'=e^{\log x\cdot y}\left(\displaystyle\frac{y}{x}+y'\log x\right)=x^{x^x}\left(x^{x-1}+e^{x\log x}(\log x+1)\log x\right)$\\

\qquad\qquad\qquad\qquad$=x^{x^x}\left[x^{x-1}+x^x(\log^2x+\log x)\right]=x^{x^x+x}\left(\displaystyle\frac{1}{x}+\log x+\log^2x\right)$\\

(2)\\

according to (1), we can get:\\

$\left(e^{x^x}\right)'=\left(e^y\right)'=e^y\cdot y'=e^{x^x}\cdot x^x(\log x+1)$\\

(3)\\

$\left(x^{e^x}\right)'=\left(e^{\log x^{e^x}}\right)'=\left(e^{e^x\log x}\right)'=e^{e^x\log x}\left(e^x\log x+\displaystyle\frac{e^x}{x}\right)$\\

(4)\\

$\left(x^{x^e}\right)'=\left(e^{\log x^{x^e}}\right)'=\left(e^{x^e\log x}\right)'=e^{x^e\log x}\left(ex^{e-1}\log x+\displaystyle\frac{x^e}{x}\right)$\\

(5)\\

$\left(x^{e^e}\right)'=e^ex^{e^e-1}$\\

(6)\\

$\left(e^{x^e}\right)'=e^{x^e}\cdot ex^{e-1}$\\

(7)\\

$\left(e^{e^x}\right)'=e^{e^x}\cdot e$\\

(8)\\

$\left(e^{e^e}\right)'=0$\\

\section{\textcolor[rgb]{0.70,0.00,0.00}{Part \uppercase\expandafter{\romannumeral2}}}(for TA)

\vspace{3.5mm}

\textcolor[rgb]{0.00,0.00,0.50}{\#1}\\

$\displaystyle\frac{d}{dx}\left(\frac{\log x}{x}\right)=\lim \limits_{h \to 0}\frac{\frac{\log(h+x)}{h+x}-\frac{\log x}{x}}{h}=\lim \limits_{h \to 0}\frac{x\log(h+x)-(h+x)\log x}{(h+x)hx}=\lim \limits_{h \to 0}\frac{x\log(\frac{h}{x}+1)-h\log x}{(h+x)hx}$\\

\qquad\qquad\qquad$\displaystyle=\lim \limits_{h \to 0}\frac{1}{(h+x)x}\cdot\frac{x}{h}\log\left(\frac{h}{x}+1\right)-\frac{\log x}{(h+x)x}=\frac{1}{x^2}\cdot1-\frac{\log x}{x^2}=\frac{1-\log x}{x^2}$\\

\textcolor[rgb]{0.00,0.00,0.50}{\#2}\\

(a)\\

since $f$ is continues, $\lim \limits_{x \to \pi}f(x)=f(\pi)$\\

and $f(\pi)=-a$, so we can get:\\

when $x<\pi$, $\displaystyle\lim \limits_{x \to \pi^-}f(x)=\lim \limits_{x \to \pi^-}\left(\frac{x^2}{2\pi}-x+a\right)=\frac{\pi}{2}-\pi+a=a-\frac{\pi}{2}$\\

when $x\geq\pi$, $\displaystyle\lim \limits_{x \to \pi^+}f(x)=\lim \limits_{x \to \pi^+}a\cos x=-a$\\

$\therefore$\qquad$-a=a-\displaystyle\frac{\pi}{2}$\\

$\therefore$\qquad$a=\displaystyle\frac{\pi}{4}$\\

(b)\\

$f'_-(\pi)=\lim \limits_{h \to 0^-}\displaystyle\frac{\frac{(\pi+h)^2}{2\pi}-(\pi+h)+a-(-a)}{h}=\lim \limits_{h \to 0^-}\frac{\frac{\pi}{2}+h+\frac{h^2}{2\pi}-\pi-h+2a}{h}=\lim \limits_{h \to 0^-}\frac{h}{2\pi}=0$\\

$f'_+(\pi)=\lim \limits_{h \to 0^+}\displaystyle\frac{a\cos(\pi+h)-(-a)}{h}=\lim \limits_{h \to 0^+}\frac{-a\cos h+a}{h}=\lim \limits_{h \to 0^+}a\frac{1-\cos h}{h}=\lim \limits_{h \to 0^+}a\frac{\sin^2\frac{h}{2}}{\frac{h^2}{4}}\cdot\frac{h}{2}=a\cdot1^2\cdot0=0$\\

$\therefore$\qquad it is differentiable at $\pi$\\

(c)\\

according to the question, we can get:\\

$f'(x)=\begin{cases}
\displaystyle\frac{x}{\pi}-1& \text{x$<$}\pi\\
\displaystyle-\frac{\pi}{4}\sin x& \text{x$\geq$}\pi
\end{cases}$\\

$ $\\

$\therefore$\qquad$f'(\pi)=0$\\

besides, $\lim \limits_{x \to \pi^-}f'(x)=\lim \limits_{x \to \pi^-}\left(\displaystyle\frac{x}{\pi}-1\right)=0$, and $\displaystyle\lim \limits_{x \to \pi^+}f'(x)=\lim \limits_{x \to \pi^+}-\frac{\pi}{4}\sin x=0$\\

$\therefore$\qquad $f'$ is continuous at $\pi$\\

\textcolor[rgb]{0.00,0.00,0.50}{\#3}\\

(a)\\

$\because$\qquad $f$ is differentiable at $a$\\

$\therefore$\qquad$\displaystyle f'_-(a)=\lim \limits_{h \to 0^-}\frac{f(a+h)-f(a)}{h}=\lim \limits_{h \to 0^+}\frac{f(a-h)-f(a)}{h}=\lim \limits_{h \to 0^+}\frac{f(a+h)-f(a)}{h}=f'_+(a)$\\

$\therefore$\qquad$\displaystyle\lim \limits_{h \to 0^+}\frac{f(a+h)-f(a)}{h}-\lim \limits_{h \to 0^+}\frac{f(a-h)-f(a)}{h}=\lim \limits_{h \to 0^+}\frac{f(a+h)-f(a-h)}{h}=0$\\

$\therefore$\qquad$\lim \limits_{h \to 0}\frac{f(a+h)-f(a-h)}{2h}=0$\\

$\therefore$\qquad it is also symmetric differentiable\\

(b)\\

$\because$\qquad$\displaystyle\lim \limits_{h \to 0}\frac{h\sin\frac{1}{h}-(-h\sin\frac{1}{-h})}{2h}=\lim \limits_{h \to 0}\frac{0}{2h}=0$\\

$\therefore$\qquad it is symmetric differentiable\\

however, $\displaystyle f'_(a)=\lim \limits_{h \to 0^-}\frac{h\sin\frac{1}{h}-0}{h}=\lim \limits_{h \to 0^-}\sin\frac{1}{h}$ and $\displaystyle\lim \limits_{h \to 0^+}\frac{h\sin\frac{1}{h}-0}{h}=\lim \limits_{h \to 0^+}\sin\frac{1}{h}$\\

And we know that, when $h$ is approaching to $0$, $\displaystyle\frac{1}{h}$ is approaching to infinity.\\

And when $\displaystyle\frac{1}{h}$ is approaching to infinity, $\sin \displaystyle\frac{1}{h}$ diverges\\

$\therefore$\qquad it is not differentiable\\

\textcolor[rgb]{0.00,0.00,0.50}{\#4}\\

$\because$\qquad $f$ is a differentiable function and $f(x+y)=f(x)+f(y)+3xy(x+y)$\\

$\therefore$\qquad$\displaystyle f'(0)=\lim \limits_{h \to 0}\frac{f(0+h)-f(0)}{h}=\lim \limits_{h \to 0}\frac{f(0)+f(h)-f(0)}{h}=\lim \limits_{h \to 0}\frac{f(h)}{h}$\\

$\therefore$\qquad$\displaystyle\lim \limits_{h \to 0}\frac{f(h)}{h}$ exists\\

besides, $\displaystyle f'(x)=\lim \limits_{h \to 0}\frac{f(x+h)-f(x)}{h}=\lim \limits_{h \to 0}\frac{f(h)+f(x)+3xh(x+h)-f(x)}{h}=\lim \limits_{h \to 0}\left[\frac{f(h)}{h}+3x(x+h)\right]$ and $\lim \limits_{h \to 0}3x(x+h)=3x^2$\\

$\therefore$\qquad$\displaystyle f'(x)=\lim \limits_{h \to 0}\left[\frac{f(h)}{h}+3x(x+h)\right]=\lim \limits_{h \to 0}\frac{f(h)}{h}+\lim \limits_{h \to 0}3x(x+h)=f'(0)+3x^2$\\

\textcolor[rgb]{0.00,0.00,0.50}{\#5}\\

(a)\\

$\because$\qquad$\displaystyle n^{[p]}\left(\frac{1}{e}\right)^n<n^p\left(\frac{1}{e}\right)^n<n^{[p]+1}\left(\frac{1}{e}\right)^n$\\

\qquad\quad and $\displaystyle\lim \limits_{n \to \infty}n^{[p]}\left(\frac{1}{e}\right)^n=\lim \limits_{n \to \infty}\left(n\cdot\left(\left(\frac{1}{e}\right)^{\frac{1}{[p]}}\right)^n\right)^{[p]}=0$\\

\qquad\quad and $\displaystyle\lim \limits_{n \to \infty}n^{[p]+1}\left(\frac{1}{e}\right)^n=\lim \limits_{n \to \infty}\left(n\cdot\left(\left(\frac{1}{e}\right)^{\frac{1}{[p]+1}}\right)^n\right)^{[p]+1}=0$\\

$\therefore$\qquad$\lim \limits_{n \to \infty}n^p\cdot e^{-n}=0$\\

(b)\\

we set $\displaystyle P\left(\frac{1}{x}\right)=a_0+a_1\left(\frac{1}{x}\right)+a_2\left(\frac{1}{x}\right)^2+\cdots+a_N\left(\frac{1}{x}\right)^N$\\

assume that $\displaystyle P(\frac{1}{x})=Q(x)\cdot\frac{1}{x^N}$\\

$\therefore$\qquad$Q(x)=a_0x^N+a_1x^{N-1}+a_2x^{N-2}+\cdots+a_N$\\

$\therefore$\qquad$\displaystyle e^{-\frac{1}{x}}P(x)=\frac{e^{-\frac{1}{x}}}{x^N}Q(x)$\\

and we can find out that $\lim \limits_{x \to 0}Q(x)=Q(0)$, and $Q(0)$ is a finite number\\

besides, $\displaystyle e^{-\left[\frac{1}{x}\right]}\left[\frac{1}{x}\right]^N\cdot\frac{1}{e}<e^{-\frac{1}{x}}\left(\frac{1}{x}\right)^N<e^{-\left(\left[\frac{1}{x}\right]+1\right)}\left(\left[\frac{1}{x}\right]+1\right)^N\cdot e$\\

and according to (a)\\

$\lim \limits_{x \to 0^+}e^{-\left[\frac{1}{x}\right]}\left[\frac{1}{x}\right]^N\cdot\frac{1}{e}=0\cdot\frac{1}{e}=0$ and $\lim \limits_{x \to 0^+}e^{-\left(\left[\frac{1}{x}\right]+1\right)}\left(\left[\frac{1}{x}\right]+1\right)^N\cdot e=e\cdot0=0$\\

$\therefore$\qquad by sandwich rule, $\lim \limits_{x \to 0^+}e^{-\frac{1}{x}}\left(\frac{1}{x}\right)^N=0$\\

$\therefore$\qquad$\lim \limits_{x \to 0^+}e^{-\frac{1}{x}}\left(\frac{1}{x}\right)^NQ(x)=\lim \limits_{x \to 0^+}e^{-\frac{1}{x}}P(\displaystyle\frac{1}{x})=0$\\

(c)\\

when $x<0$, $f^{(n)}(x)=0$ for all $n$, so it is differentiable\\

when $x>0$, by induction we know:\\

$\displaystyle f'(x)=\frac{1}{x^2}e^{-\frac{1}{x}}=e^{-\frac{1}{x}}P_1(\frac{1}{x})$\\

$\displaystyle f^{(2)}(x)=e^{\frac{1}{x}}\frac{1}{x^2}P_1(x)+e^{\frac{1}{x}}P'_1(\frac{1}{x})=e^{-\frac{1}{x}}P_2(\frac{1}{x})$\\

$\cdots$\\

$\therefore$\qquad$\displaystyle f^{(n)}(x)=e^{\frac{1}{x}}P(\frac{1}{x})$\\

so it is differentiable\\

when $x=0$\\

$f^{(n)}_-(x)=\lim \limits_{h \to 0}\displaystyle\frac{0-0}{h}=0$, $f^{(n)}_+(x)=\lim \limits_{h \to 0^+}\frac{f^{(n-1)}_+(0+h)-f^{(n)}_+(0)}{h}=\lim \limits_{h \to 0^+}e^{\frac{1}{x}}P(\frac{1}{x})=0$\\

so it is differentiable at $0$\\

so it is infinitely many times differentiable everywhere\\

\end{document}
